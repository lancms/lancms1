\chapter{After installation - a quick intro to using lancms}

After installing the software, log in as \lstinline!globaladmin / admin! and
create your first event in globaladmin. After creating the event you should go
back to globaladmin and set it as public, so that normal users may find it.

Since lancms has support for multiple events and multiple URLs, you have to go
to \lstinline!globaladmin! and then \lstinline!change global options! and set default eventID
for the hostname you're using for your installation of lancms.
'1' is the default id, no event, event. The first event you create get id '2'.

\section{Adding tickets}

You probably need to add some tickets so that your attendees can buy access to
your event. In \lstinline!eventadmin!, go to \lstinline!ticketadmin! and add a
ticket type. Set a name, the price of the ticket and what type of ticket it
is. You can choose between:
\begin{itemize}
\item prepaid - an admin need to acknowledge that the ticket has been paid
for before you can choose a seat
\item preordered - you select a seat before you pay, payment is done when you
arrive
\item onsite with computer - you haven't ordered a ticket before you show up
in the door
\item onsite without computer - visitor
\item reseller - you pay in store to get a code for a ticket
\end{itemize}

Prepaid and preordered tickets can be ordered directly by users, but onsite
tickets has to be assigned to the user in the arrival-module.

\section{Desiging a seatmap}

It's nice for the attendees to know where they are going to sit.
In seatadmin you can design how you set up your floorplan and choose
different types of seats/tiles:
\begin{itemize}
\item Wall / Door
\item Open seat - normal seat, can be chosen by anyone, at once on preordered
or after payment is recieved on prepaid
\item Group - can be chosen based on what group an user is member of. Useful
to assign seats to a group (clan or crew)
\item Password - assign a password to the seats. Lets you give users access to
seats with having to be members of a specific group
\item Text - putting text on the map (FIXME. Doesn't work at the moment)
\item Area - placing areas on the map, for example "check-in" or "kiosk" (FIXME. Doesn't work at the moment)
\end{itemize}

Before you can start designing the seatmap, you have to reset it to create it
and setup the event to have a seatmap.

\section{Starting sales}

In \lstinline!eventconfig! under \lstinline!eventadmin! you can enable
different modules. To allow users to buy tickets must enable
\lstinline!ticketorder!. To allow users to pick seats, you enable
\lstinline!seating!, etc.

\section{Giving out rights}

You'll have to give people rights to allow them to access the
administration parts of the system.
In \lstinline!group management! under \lstinline!eventadmin! you can create
different groups or crews. Create a group, for example "Securitycrew", and go
to \lstinline!Change group rights! If "Securitycrew" is meant to work with
arrivals and recieve payment for tickets they would need access to
tickets (with \lstinline!ticketsadmin!), changing seats for users (with
\lstinline!seating!) and correcting user information (with
\lstinline!userAdmin!). An access of "Read" gives a right to see information,
"Write" enables the users to most tasks (for example changing tickets assigned
to users etc.) and "Admin" gives access to change everything, including adding
new ticket types etc.
