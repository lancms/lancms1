\chapter{The modules - advanced usage of lancms}

\section{Globaladmin}

Globaladmin is the module where you changes that affect the entire lancms
installation. You can create and change events, view the logs and modify
users.

Only users with global admin-rights can see \textit{globaladmin}. This
right can normally not be given out, but there are two ways of giving it to
more than the first "globaladmin"-user:
\begin{itemize}
\item By finding the user you want to give global admin-rights to in MySQL
(phpmyadmin or other tool) and changing the field globaladmin to
'1' from '0'.
\item By putting the following in your config.php or OverrideConfig.php: \textit{\$globalaccess[]='globaladmin';}
Afterwards you have to create an
accessgroup in an event and give it the globaladmin-right. You should for
security-reasons remove the globalaccess-line from your config afterwards.
\end{itemize}

Events can have two operating modes: public and private. If the event is
public, anyone can see it in the eventlist (in the menu). If it's private, you
have to be given permission to see it (or have global admin-rights). To give
someone access to a private event, set that event as active and go to
\textit{Attendee-access}. That will give you a list of all access-groups in
all events and you can press \textit{allow access} to give the members of
that group access to the event.

In \textit{Global options} you can configure what information us required when users register, as well as enable registration and creation of clans.
You can also define which events a users should see as default when they visit your webserver, by changing the \textit{hostname\_your\_webservers\_hostname} to the eventID of your event. The default is '1', which means 'no event'.

%\section{Eventadmin}

%\section{Composystem}

\section{Arrival}
In \textit{arrival} you can change the status of each ticket and user. Users with \textit{ticketadmin}-rights (Write or Admin) automaticly gets access to \textit{arrival}.
You can search for the users name (either firstname or lastname, not both), nickname or email. When you've found the correct user, you will see the tickets that are assigned (only the ones that he is the User of) to that user. If the background is orange, the ticket is not paid for. If it's green, it's paid.
Selecting the ticket name will show you more information about the user and the ticket. You should check that the users information is correct if you need to use the information later on, for example for memberships.
Below the user information is the ticket information. The first cell marks if the ticket has been paid or not, and the price of the ticket.
The second cell shows if the ticket have a seat on the seatmap assigned. The third cell is for deleting the ticket permanently.
